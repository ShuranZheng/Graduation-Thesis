
\chapter{算法分析}
\label{cha:algo}
在这章中,我们给出一个可以与下界比较且较优的算法,并对正确性和采样复杂度进行分析。
\section{逐步排除算法}
  再给出算法之前,我们先定义一个函数$LowAll(S, \varepsilon, \delta)$,这里$S$是一个$n$维空间中的点集,$\varepsilon$和$\delta$是两个常数,这个函数的返回值是一个长度为$n$的向量,表示了我们至少需要测量每个$c_i$多少次, 才能以至少$1-\delta$的概率、$\varepsilon$的精度,估计$S$中任意一对点之间的目标函数值的差. 定义函数$LowAll(S, \varepsilon, \delta)$为:
        \begin{align*}
        \min_{\tau} \quad & \sum_{i=1}^n \tau_i\\
        s.t.\quad & \sum_{i=1}^n \frac{(x_i -y_i)^2}{\tau_i} \le \frac{\varepsilon^2}{2 \ln(2/\delta)}, \forall x, y \in S\\
         &\tau_i\ge 0, \forall i.
        \end{align*}
我们的逐步排除算法在一开始将所有可能的最优解(即可行域的所有顶点)纳入考虑范围,每一次循环中算法都会根据$LowAll(S, \varepsilon, \delta)$的值对每个$c_i$进行测量,再根据测量结果删除掉与最优解的差大于某一个值的点,每一次循环中算法都会将精度$\varepsilon$减半,从而能够排除掉更多的点。当只剩下一个未被排除的点时,算法结束。
	\begin{algorithm}[H]                      % enter the algorithm environment
                \caption{逐步排除算法}
                \label{Algo1}
                \begin{algorithmic}[1]
                	\REQUIRE 线性规划问题$\mathcal{I} = \max_{Ax\le b} c^T x$.
                    \STATE $S^1\gets$所有基本可行解的集合,或者说可行域$\{x: Ax\le b\}$所有顶点的集合
                    \STATE $r\gets 1$
                    \STATE $\lambda\gets 10$
                    \WHILE{$|S^r|>1$}
                        \STATE $\varepsilon^r \gets 2^{-r}$, $\delta^r \gets \delta/(10 r^2 |S^1|^2)$
                        \STATE $(m^r_1, \dots, m^r_n) \gets LowAll(S^r, \varepsilon^r/\lambda, \delta^r)$
                        \STATE 对$c_i$进行$m^r_i$次测量, 令$\hat{c}^r_i$为这$m^r_i$次测量的均值
                        \STATE 令$x^r$为当目标函数是$\hat{c}^r$时,集合$S^r$中的最优值,即
                        $x^r = \arg \max_{x\in S^r} x^T \hat{c}^r$.
                        \STATE 去除$S^r$中我们认为在目标函数为$\hat{c}^r$时,比$x^r$要差至少$\varepsilon^r/2+2\varepsilon^r/\lambda$的点,即$S^{r+1} \gets \{ x\in S^r: x^T \hat{c}^r \ge (x^r)^T \hat{c}^r-\varepsilon^r/2-2\varepsilon^r/\lambda\}$
                        \STATE $r \gets r+1$
                    \ENDWHILE
                    \STATE 输出$S^r$中仅剩的一个点。
                \end{algorithmic}
                \end{algorithm}

    \begin{theorem}
    令$\mathcal{I}$为一个目标函数未知的线性规划问题$\max_{\{x:Ax\le b\}} c^T x$. 有至少$1-\delta$的概率, 逐步排除算法会输出正确的最优解, 且使用的测量次数为
     \[
        O\left( Low(\mathcal{I}) \ln \Delta^{-1} ( n + \ln \delta^{-1} + \ln \ln \Delta^{-1})\right),
     \]
     其中$n$是线性规划变量的个数, $\Delta$是目标函数为$c$时,最优的基本可行解与次优的基本可行解之间的差值,即
     \[
     \Delta = \max_{x\in S^1} c^T x - \max_{x\in S^1 - x^*} c^T x.
     \]
    \end{theorem}
	
\section{算法正确性}
 在这一节中,我们证明该算法是$\delta$-正确算法。在证明中我们会用到下面这个引理,
    \begin{lemma}\label{lem1}
 考虑目标函数$c = (c_1, \dots, c_n)$, 假设我们对$c_i$进行$\tau_i$次测量,每一次测量的误差服从标准高斯分布,令$X_i$为这$\tau_i$次测量的平均值。那么对任意$n$维向量$p$, 有 
    \[
    \Pr\left[ |p^T X - p^T c|\ge \varepsilon\right] \le 2 \exp\{-\frac{\varepsilon^2}{2 \sum p_i^2/\tau_i}\}
    \]
    \end{lemma}
    \begin{proof}
    由定义可知, $p^T X - p^T c$服从均值为$0$方差为$\sum_i p_i^2/\tau_i$的高斯分布,由高斯分布的tail bound即可得到上述不等式. 
    \end{proof}
定义事件$\mathcal{E}$为下述条件成立的事件$$|(x-y)^T (\hat{c}^r - c))|\le \varepsilon^r /\lambda,\ \forall r, \forall x,y \in S^r.$$ 
根据引理~\ref{lem1}和union bound, $\Pr[\mathcal{E}]\ge 1-\delta$. \tbd{prove this?} 那么我们只需证明以下引理,
\begin{lemma}\label{correct}
在事件$\mathcal{E}$成立的时候,最优解$x^* = \max_{Ax\le b} c^T x$不会被算法删除。
\end{lemma}
\begin{proof}
假设$x^*$在第$r$个循环中被删除,即$x^* \in S^r$, $x^* \notin S^{r+1}$,那么我们有
\[
\langle \hat{c}^r, x^*\rangle < \langle \hat{c}^r, x^r\rangle-\varepsilon^r/2-2\varepsilon^r/\lambda.
\]
即
\[
\langle \hat{c}^r, x^* - x^r\rangle < -\varepsilon^r/2-2\varepsilon^r/\lambda.
\]
而由$x^*$的定义知
\[
\langle c, x^* - x^r\rangle > 0.
\]
结合两个不等式,我们可以得到
\[
\langle c - \hat{c}^r, x^* - x^r\rangle > \varepsilon^r/2 + 2\varepsilon^r/\lambda > \varepsilon^r /\lambda,
\]
由于$x^*$和$x^r$都是$S^r$中元素,这与$\mathcal{E}$成立矛盾。故当$\mathcal{E}$成立,最优解$x^*$不会被算法删除。
\end{proof}
\section{采样复杂度分析}

我们再对算法的采样复杂度进行分析。与上一节中一样,定义事件$\mathcal{E}$为下述条件成立的事件$$|(x-y)^T (\hat{c}^r - c))|\le \varepsilon^r /\lambda,\ \forall r, \forall x,y \in S^r.$$ 
根据引理~\ref{lem1}和union bound, $\Pr[\mathcal{E}]\ge 1-\delta$。

我们首先给出算法循环次数的上界,
\begin{lemma}\label{itrnum}
当$\mathcal{E}$成立时,算法循环次数不会超过$\lfloor \log(\Delta^{-1})\rfloor + 1$,其中$\Delta$是当目标函数为$c$时,最优的基本可行解与次优的基本可行解之间的差值,即
     \[
     \Delta = \max_{x\in S^1} c^T x - \max_{x\in S^1 - x^*} c^T x.
     \]
\end{lemma}
\begin{proof}
我们只需证明在第$r$次循环结束后,$S^{r+1}$中所有的元素$s$都满足以下条件,
\[
\langle c, x^* - s\rangle < \varepsilon^r, \ \forall s\in S^{r+1}.
\]
假设在进入第$r$次循环时,存在$s\in S^r$使得
\[
\langle c, x^* - s\rangle > \varepsilon^r,
\]
那么由于$\mathcal{E}$成立,且我们选择$\lambda=10$,
\[
\langle c^r, x^* - s\rangle > \langle c, x^* - s\rangle - \varepsilon^r/\lambda > (1-1/\lambda)\varepsilon^r  >\varepsilon^r/2 + 2\varepsilon^r/\lambda,
\]
又因为引理\ref{correct},$x^*$一定存在于$S^r$中。故$s$会在第$r$个循环中被删除。
\end{proof}
下面我们给出算法在第$r$次循环中使用的测量次数的上界。令$n$维向量$\tau$为定理\ref{inslow}中定义的$Low(\mathcal{I})$的返回值. 考虑第$r$次循环, 定义$$\alpha = 32 \lambda^2 \ln(2/\delta^r).$$ 我们证明$m = \alpha \tau^*$满足$LowAll(S^r, \varepsilon^r, \delta^r)$所有的约束条件. 对于任意的$x,y\in S^r$, 我们有 
\begin{align*}
\sum \frac{(x_i-y_i)^2}{m_i} &= \frac{1}{\alpha} \sum \frac{(x_i-y_i)^2}{\tau_i^*}\\
& = \frac{1}{\alpha} \sum \frac{(x_i-x^*_i + x^*_i - y_i)^2}{\tau_i^*}\\
& \le \frac{1}{\alpha} \sum \frac{2(x_i-x^*_i)^2 + 2(x^*_i - y_i)^2}{\tau_i^*}\\
& \le \frac{2}{\alpha} \left((c^T(x^*-x))^2 + (c^T(x^* - y))^2\right)\\
& \le \frac{4}{\alpha} (\varepsilon^{r-1})^2 = \frac{(\varepsilon^r)^2}{2\lambda^2 \ln(2/\delta^r)}.
\end{align*}
在第四行中,我们用到了$Low$函数\ref{LowGeneral}的定义,在第五行中我们用到了引理\ref{itrnum}证明中的结论。

所以在第$r$次循环中,我们使用的测量次数不会超过
\[
\sum_i m_i = \alpha \sum_i \tau_i^* = O(Low(\mathcal{I})(\ln r + \ln |S^r| + \ln \delta^{-1}).
\]
结合引理\ref{itrnum},循环次数不会超过$\lfloor \log(\Delta^{-1})\rfloor + 1$,故算法使用的总测量数不会超过
\[
O\left( Low(\mathcal{I}) \ln \Delta^{-1} ( n + \ln \delta^{-1} + \ln \ln \Delta^{-1})\right).
\]
