\chapter{问题描述}
\section{目标函数未知的线性规划问题}
假设我们手上有一个线性规划问题,
\begin{align*}
\max &\ \ c^T x, \\
\textrm{ s.t. }& \ \ Ax \le b.
\end{align*}
我们想要找到这个问题的最优解,但是不知道这个问题里的目标函数,即向量$c$。假设我们可以通过一些方法去测量向量$c$里的每一个参数$c_i$,但是测量值$\tilde{c_i}$可能会有随机误差,假设这个随机误差服从均值为$0$,方差为$\sigma_i^2$的正态分布,即测量值$\tilde{c_i}$服从均值为$c_i$,方差为$\sigma^2$的正态分布。我们的目标是找到一个算法,能够保证有很高的概率可以找到正确的最优值,且使用的测量次数尽量少。

在这片论文中,我们假设对所有的$i$都有$\sigma_i=1$。在本文介绍的所有结论中,方差$\sigma_i\neq 1$的情形都可以由$\sigma_i=1$的情形通过简单推导得到;或者也可以把多次测量值的平均值当作一次测量,这样就可以直接使用本文中的结论。

下面我们举几个具体的例子。
\begin{example}[最短路问题]
假设我们想要求在图$E=(U,V)$上节点$s$到节点$t$的最短路,我们不知道每条边的边权,但可以每次选一条边去测量。这个问题就可以被描述成如下目标函数未知的线性规划问题
\begin{align*}
\max &\ \ -\sum_{(u,v)\in E} w_{uv} x_{uv},\\
\textrm{ s.t. }& \ \ \sum_v x_{uv} - \sum_v x_{vu} = 0, \ \forall u,\\
& \ \ x_{uv}\ge 0, \forall u,v.
\end{align*}
其中$E$为边集,$w_{uv}$为边权。
\end{example}

\begin{example}[最大流问题]
假设我们想求在图$E=(U,V)$上从$s$到$t$的最大流,但是我们不知道每条边的流量上限,但可以每次选一条边去测量。这个问题就可以被描述成如下目标函数未知的线性规划问题
\begin{align*}
\max &\ \ -\sum_{(u,v)\in E} c_{uv}x_{uv}, \\
\textrm{ s.t. }& \ \  y_v + y_{sv} \ge 1, \ \ \forall v: (s,v)\in E,\\
& \ \ y_v -y_u +y_{uv} \ge 0, \ \ \forall(u,v)\in E, u\neq s, v\neq t,\\
& \ \ -y_u + y_{ut}\ge 0, \ \ \forall u: (u,t)\in E.
\end{align*}
其中$E$为边集,$c_{uv}$为边的流量上限。
\end{example}

\section{$\delta$-正确算法}
由于随机误差的存在,我们只能期望以高概率找到正确的线性规划最优解。
\begin{definition}[$\delta$-正确算法]
一个算法$\mathcal{A}$是$\delta$-正确算法,如果对于任意目标函数未知的线性规划问题,算法$\mathcal{A}$输出正确最优解$x^* = \max_{Ax\le b} c^T x$的概率至少为$1-\delta$.
\end{definition}

\tbd{give an example of $\delta$-correct algorithm?}